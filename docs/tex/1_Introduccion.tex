\capitulo{1}{Introducción}

Lo primero de todo, quiero agradecer a mis tutores por la paciencia que han tenido conmigo. Desde el inicio del desarrollo del proyecto me han explicado de la mejor forma posible, aportando fuentes de información e ideas. Quiero dar las gracias a Bruno, Alfredo y Héctor, no sé si este proyecto hubiera podido salir hacia adelante sin vosotros.

\noindent\rule{2cm}{0.4pt}

En este proyecto se ha desarrollado una aplicación web que nos proporcione información (noticias, meteorología, eventos, calidad del aire, etc.) de una ciudad o un punto geográfico de interés para el usuario a tiempo real. Además, esta aplicación le permite al usuario planificar sus actividades al aire libre o hacer planes de viaje con anticipación.

Actualmente, existen aplicaciones web de noticias, meteorología, eventos, calidad del aire, etc. solo que estas aplicaciones permiten consultar una sola categoría de información. No hay muchas aplicaciones web que nos permitan consultar la información de noticias, meteorología, eventos, calidad del aire, etc. de una ciudad dentro de la misma aplicación.

\project{ANME} tiene el propósito de crear una aplicación web nos proporcione información de una ubicación de España a tiempo real usando diferentes APIs.
Para esto se ha utilizado el tipo de integración de datos ETL (Extract,Transform,Load).

Aunque no sea la única aplicación web que proporcione información de noticias, eventos y meteorología, el proyecto me ha permitido aprender de integración de APIs y ha sido un ejercicio de aprendizaje para iniciarme en el mundo del desarrollo web.

\newpage
\section{Estructura de la memoria}

\begin{itemize}
    \item \textbf{Objetivos del proyecto}: En esta sección vamos a enumerar los objetivos que proponemos lograr en el desarrollo de este proyecto.
    \item \textbf{Conceptos teóricos}: En este apartado vamos a explicar los conceptos teóricos claves para poder entender la solución propuesta.
    \item \textbf{Técnicas y herramientas}: En este apartado vamos a enumerar y a explicar todas las herramientas utilizadas en la realización del proyecto ANME. Además, explicaremos como las hemos utilizado nosotros. 
    \item \textbf{Aspectos relevantes del desarrollo}: En esta sección destacaremos los aspectos más relevantes del desarrollo llevado a cabo.
    \item \textbf{Trabajos relacionados}: En este apartado se pretende mostrar proyectos similares a ANME disponibles en el mercado.
    \item \textbf{Conclusiones y líneas futuras}: En este apartado dialogaremos las conclusiones que hemos obtenido al finalizar el proyecto y revisaremos si se han cumplido todos los objetivos previstos. 
\end{itemize}

