\apendice{Documentación de usuario}

\section{Introducción}
En el siguiente manual, se describen los pasos que un usuario debe seguir para utilizar nuestra aplicación web de manera sencilla y eficaz.

\section{Requisitos de usuarios}

A continuación, se enumeran los requisitos indispensables para poder usar la aplicación \textbf{ANME}:

\begin{itemize}
    \item Disponer de un dispositivo con un navegador web (por ejemplo Chrome, Firefox, etc.).
    \item La aplicación \textbf{ANME} no funciona offline, por lo que se necesita una conexión a internet para hacer uso de la misma.
    \item La aplicación \textbf{ANME} es accesible desde cualquier dispositivo, pero se recomienda ser usada en un ordenador para tener una experiencia más óptima.
\end{itemize}

\section{Instalación}

En nuestro caso, no es necesario que el usuario instale ninguna herramienta para poder usar la aplicación. 

\section{Manual del usuario}

Podemos acceder a la aplicación \textbf{ANME} desde \url{https://www.anme.city}.El usuario necesitará registrarse para hacer uso de la misma.

\subsection{Inicio}

Una vez que accedemos a la web de la aplicación veremos en la pantalla de nuestro dispositivo  una ventana como la que aparece en la siguiente figura:

\imagen{Inicio}{Página de Inicio}

En esta pantalla el usuario debe iniciar sesión con el usuario y contraseña con la que se ha registrado en la aplicación. Si el usuario no se ha registrado, debe crearse una cuenta en el botón superior izquierdo llamado \textbf{Registro}, el cual nos redirigirá a la ventana de registro.

En caso de que el campo Usuario o Contraseña sean incorrectos, se mostrará una notificación diciendo que el usuario o la contraseña son incorrectos.
\newpage
\subsection{Registro}

\imagen{Registro}{Página de Registro}

Tras haber hecho clic en el botón  de \textbf{Registro} en la ventana de Inicio, este nos redirige a la ventana de Registro en la cual podremos registrarnos. Para registrarnos en la aplicación es necesario rellenar los campos que nos aparecen en la ventana de Registro, que son: Nombre, Usuario y Contraseña.

Una vez que se han rellenado todos los campos de la ventana Registro hacemos clic en el botón \textit{Register}, si todos los campos son válidos el usuario se ha registrado exitosamente en la aplicación y se le redirigirá a la ventana de Inicio donde iniciará sesión. En caso contrario, se mostrará una notificación de error.

\subsection{Barra de Navegación}

\imagen{BarraNavegacion}{Barra de Navegación}

La \textit{barra de navegación} es el elemento que va a permitir al usuario poder navegar por la aplicación. Está formada por seis botones.

A continuación se detallará la funcionalidad de cada uno de los botones:

\begin{itemize}
    \item Botón \textit{Inicio}: Permite al usuario navegar a la ventana principal de la aplicación.
    \item Botón \textit{Noticias}: Permite al usuario navegar a la ventana de Noticias en la que se muestran noticias de forma general de España.
    \item Botón \textit{Eventos}: Este botón nos redirigirá a la pantalla de Eventos en la que se muestran todos los eventos disponibles en España.
    \item Botón \textit{Meterología}: Este botón nos redirigirá a la pantalla de Meterología en la que se muestra el tiempo actual y una previsión de 6 días de la ubicación actual del usuario.
    \item Botón \textit{Eventos Favoritos}: Este botón nos redirigirá a la pantalla de Eventos Favoritos en la que se muestran todos los eventos añadidos a favoritos por el usuario.
    \item Botón \textit{Cerrar sesión}: Este botón permitirá al usuario cerrar su sesión de la aplicación.
\end{itemize}

\subsection{Ventana Principal}

\imagen{Principal}{Página principal}

Tras haber iniciado sesión en la aplicación, la web nos redirige a la ventana  Principal, donde nos aparece una barra de navegación y una breve descripción de los servicios que nos proporciona la misma.

Desde esta ventana el usuario ya puede acceder a los diferentes servicios que nos proporciona esta aplicación mediante la barra de navegación que aparece en esta ventana.

\newpage
\subsection{Noticias}

\imagen{Noticias}{Página de Noticias}

En esta ventana se visualizan un gran número de  noticias que han sido publicadas en España. Además, nos permite realizar un filtrado de noticias tanto por ubicación como por categoría en caso de que fuese de interés para el usuario.

Si el usuario deseara leer una noticia de la presente ventana, deberá hacer clic en el botón \textit{Ver Noticia}, y este será redirigido a la noticia que ha solicitado leer.

Si se desea realizar una búsqueda de noticias por ubicación, hay que introducir la ubicación que deseemos en el filtro \textit{Ubicación} y le daremos al botón de \textit{Buscar}.

\imagen{NoticiasUbic}{Buscar Noticias por ubicación}

Si se desea realizar una búsqueda de noticias por categoría, hay que seleccionar la categoría que deseemos en el filtro \textit{Categorías} y le daremos al botón de \textit{Buscar}.

\imagen{NoticiasCateg}{Buscar Noticias por categoría}

También contamos con una barra de navegación en esta ventana desde la cual el usuario se le permite acceder a cualquier servicio de la aplicación.


\subsection{Eventos}

\imagen{Eventos}{Página de Eventos}

En esta ventana se visualizan eventos de diferentes categorías, como eventos de música, deportes, películas, que están disponibles en toda España. Además, nos permite realizar un filtrado de eventos tanto por ubicación como por categoría en caso de que fuese de interés para el usuario.

Si el usuario deseara ver un evento de la presente ventana, deberá hacer clic en el botón \textit{Ver Evento}, y este será redirigido a la página de información del propio evento.

Si se desea realizar una búsqueda de eventos por ubicación, hay que introducir la ubicación que deseemos en el filtro \textit{Ubicación} y le daremos al botón de \textit{Buscar}.

\imagen{EventosUbic}{Buscar eventoss por ubicación}

Si se desea realizar una búsqueda de eventos por categoría, hay que seleccionar la categoría que deseemos en el filtro \textit{Categorías} y le daremos al botón de \textit{Buscar}.

\imagen{EventosCateg}{Buscar eventoss por categoría}

También contamos con una barra de navegación en esta ventana desde la cual el usuario se le permite acceder a cualquier servicio de la aplicación.

\imagen{BarraNavegacion}{Barra de Navegación}

\newpage
\subsection{Ventana Información del Evento}

\imagen{InfoEventos}{Página de Informacón del Evento}

Una vez que se ha pulsado el botón \textit{Ver Evento} en la página de eventos, el sitio web redirige al usuario a la ventana de información del evento donde se muestran las características específicas del evento solicitado.

Si el usuario desea guardar un evento debido a su interés en él, simplemente debe hacer clic en el botón 'Añadir a favoritos', y el evento se guardará automáticamente en la lista de eventos favoritos del usuario.

\newpage
\subsection{Meteorología}

\imagen{Tiempo}{Página de Meteorología}
\imagen{Prevision}{Página de Meteorología}

En esta ventana, se presenta la información meteorológica de la ubicación actual del usuario, donde el usuario puede conocer las características del tiempo actual. Además, el usuario dispone de la previsión del tiempo para una ubicación específica durante los próximos seis días. Esto le permite al usuario planificar sus actividades al aire libre o hacer planes de viaje con anticipación.

Si el usuario decide consultar la meteorología para una ubicación diferente a la actual, el usuario deberá ingresar la ubicación deseada en la barra de búsqueda de la ventana actual y presionará el botón "Buscar". De esta manera, el usuario podrá visualizar la información del clima para la ubicación especificada. 

\newpage
\subsection{Eventos Favoritos}
En esta ventana el usuario podrá visualizar los eventos que ha añadido a su lista de favoritos. Además, para cada evento se visualiza la información meteorológica de la ciudad donde se llevará a cabo el propio evento. Esto le permite al usuario anticiparse.

Si el usuario lo desea, puede visualizar todos los eventos guardados en favoritos en un mapa para que  pueda ubicarlos de manera sencilla.

Si un evento deja de ser de interés para el usuario, puede eliminarlo de la lista de favoritos pulsando el botón \textit{Eliminar de Favoritos}.

\imagen{EventosFav}{Lista de Eventos Favoritos}
\imagen{TiempoEventosFav}{Meteorología de Eventos Favoritos}
\imagen{MapaEventosFav}{Mapa de Eventos Favoritos}

